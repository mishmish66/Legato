\documentclass[12pt, twocolumn]{article}
\usepackage[margin=1in]{geometry}
\usepackage{amsmath,amssymb,amsthm}
\usepackage{graphicx}
\usepackage{hyperref}

\begin{document}
\title{LEGATO: Latent Embedding Gradient Aware Trajectory Optimization}
\maketitle
\begin{abstract}
Previous MPC techniques rely on a simplified handcrafted model for optimizing control.T
These handcrafted models 
\end{abstract}
\section{Introduction}
Introduce the problem of learning latent space models for reinforcement learning. Provide motivation and background. Explain the key challenges and contributions of your work.
\section{Related Work}
Discuss prior work on latent space models, model-based RL, and other relevant topics. Explain how your approach differs from or builds upon previous methods.
\section{Approach}
Describe your overall approach in detail. Explain the key components:
\begin{itemize}
    \item State/action encoders and decoders
    \item Latent transition model
    \item Loss functions (transition loss, smoothness loss, coverage loss)
    \item Actor policy
\end{itemize}
Use equations, pseudocode, and figures as needed to clarify the technical details.
\section{Experiments}
Describe your experimental setup, including the environment/data used. Present quantitative and qualitative results demonstrating the performance of your approach. Use tables and plots to visualize the results. Provide analysis and insights.
\section{Conclusion}
Summarize the key findings and contributions. Discuss limitations and directions for future work.
\begin{thebibliography}{99}
    \bibitem{} References here
\end{thebibliography}
\appendix
\section{Implementation Details}
Include any additional implementation details, hyperparameter settings, etc. that would help in reproducing the results.
\section{Additional Results}
Include any supplementary results or analysis that didn't fit in the main paper.
% \lstinputlisting[language=Python, caption=Key Code Snippets]{code.py}
\end{document}